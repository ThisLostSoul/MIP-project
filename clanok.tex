% Metódy inžinierskej práce

Here is the translation of the document from Slovak to English:

```latex
\documentclass[10pt,twoside,english,a4paper]{article}

\usepackage[english]{babel}
%\usepackage[T1]{fontenc}
\usepackage[IL2]{fontenc} % better typesetting for the letter Ľ than in T1
\usepackage[utf8]{inputenc}
\usepackage{graphicx}
\usepackage{url} % the \url command for formatting URLs
\usepackage{hyperref} % the links in the text will be active (for some document classes, this may cause text shift)

\usepackage{cite}
%\usepackage{times}

\pagestyle{headings}

\title{Title\thanks{Semester project in the course Engineering Methods, academic year 2015/16, supervision: Firstname Lastname}} % name and surname of the instructor in the exercises

\author{Firstname Lastname\\[2pt]
	{\small Slovak University of Technology in Bratislava}\\
	{\small Faculty of Informatics and Information Technologies}\\
	{\small \texttt{...@stuba.sk}}
	}

\date{\small 30 September 2015} % adjust



\begin{document}

\maketitle

\begin{abstract}
\ldots
\end{abstract}



\section{Introduction}

Motivate the reader and explain what you are writing about. The introduction is usually not divided into sections.

Explicitly mention the structure of the article. Here is an example.
The basic problem hinted at in the introduction is explained in more detail in Section~\ref{some}.
Important context is provided in Sections~\ref{important} and~\ref{more_important}.
Final remarks are provided in Section~\ref{conclusion}.



\section{Some Section} \label{some}

Everything is clear from Fig.~\ref{f:decision}.

\begin{figure*}[tbh]
\centering
%\includegraphics[scale=1.0]{diagram.pdf}
Even text can be presented as a figure. It becomes a labeled floating object. After creating a diagram, remove the \texttt{\%} symbol in front of the \verb|\includegraphics| command and mark this line as a comment (also using the \texttt{\%} symbol).
\caption{The decisive argument.}
\label{f:decision}
\end{figure*}



\section{Another Section} \label{another}

The basic problem is therefore\ldots{} First, let's look at some explanation (Section~\ref{another:some}), and then at some more explanation (Section~\ref{another:more}).\footnote{Sometimes you may need a footnote.}

It may seem that the problem does not actually exist\cite{Coplien:MPD}, but it has been proven otherwise~\cite{Czarnecki:Staged, Czarnecki:Progress}. Nevertheless, even today, we encounter all kinds of dubious opinions on the web\cite{PLP-Framework}. Important things can be \emph{emphasized in italics}.


\subsection{Some Explanation} \label{another:some}

Sometimes, a list needs to be provided:

\begin{itemize}
\item one thing
\item another thing
	\begin{itemize}
	\item x
	\item y
	\end{itemize}
\end{itemize}

The same list, but numbered:

\begin{enumerate}
\item one thing
\item another thing
	\begin{enumerate}
	\item x
	\item y
	\end{enumerate}
\end{enumerate}


\subsection{More Explanation} \label{another:more}

\paragraph{A very important note.}
Sometimes a heading is needed to mark a paragraph. The text continues right after the heading.



\section{Important Section} \label{important}




\section{Even More Important Section} \label{more_important}




\section{Conclusion} \label{conclusion} % or another variation of the title



%\acknowledgement{If you want to thank someone\ldots}


% this generates the bibliography list from the contents of the literatura.bib file based on the citations in the article
\bibliography{literatura}
\bibliographystyle{plain} % alternatively, alpha, abbrv, or any other style
\end{document}
```